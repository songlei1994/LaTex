\documentclass[a4paper]{ctexart}
\newcommand{\ts}[2]{#1\otimes #2} 
\newcommand{\tss}[3]{#1\otimes_{#2} #3} 
\newcommand{\es}[5]{$#1\xrightarrow{#2}#3\xrightarrow{#4}#5\xrightarrow{}0$} 
\newcommand{\ess}[5]{$0\xrightarrow{}#1\xrightarrow{#2}#3\xrightarrow{#4}#5\xrightarrow{}0$}
\RequirePackage{amsmath,amsthm,amsfonts,amssymb,bm,mathrsfs,wasysym}
\RequirePackage{fancyhdr}
\newsavebox{\mygraphic}
\sbox{\mygraphic}{\includegraphics[totalheight=1cm]{1.eps}}
\fancypagestyle{plain}{
\fancyhf{}
\fancyhead[LE]{\usebox{\mygraphic}\zihao{-4}~同调论~\today}
\fancyhead[LO]{\usebox{\mygraphic}\zihao{-4}~同调论~\today}
\fancyhead[RO,RE]{\zihao{-4} 宋雷~1601210073}
\fancyfoot[C]{\small -~\thepage~-}}
\RequirePackage[top=2cm,bottom=2cm,left=1cm,right=1cm]{geometry}
\renewcommand{\baselinestretch}{1.5}
\usepackage{extarrows}
\usepackage{exscale}
\usepackage{tikz} 
\usepackage{relsize}
%\usepackage{fourier} 
\begin{document}
\pagestyle{plain}
\noindent
\zihao{4}
\\
1.一个流形上的所有联络的集合是一个凸集,即如果$D^1,\cdots ,D^k$节食联络,又若$f_1,\cdots ,f_k$是$C^\infty$函数,具有性质
\[\sum_{i=1}^kf_k=1,\]
那么$\sum\limits_{i=1}^kf_k\mathrm{D}^k$也是一个联络.\\
首先联络之间的加法,以及与函数的乘法使用常用的方法定义.验证$C1-C3$即可.\\
C1:
\begin{align*}
\left( \sum_{i=1}^kf_k\mathrm{D}^k\right) _{fV+gW}X 
&:=\sum\limits_{i=1}^kf_k\mathrm{D}^k _{fV+gW}X\\
&\overset{C1}{=}\sum\limits_{i=1}^kf_k\left(f\mathrm{D}^k _{V}X+g\mathrm{D}^k _{W}X \right)\\
& =\sum_{i=1}^kf_kf\mathrm{D}^k _{V}X+\sum_{i=1}^kf_kg\mathrm{D}^k _{W}X\\
&=f\left( \sum_{i=1}^kf_k\mathrm{D}^k _{V}X\right) +g\left( \sum_{i=1}^kf_k\mathrm{D}^k _{W}X\right) \\
&:=f\left(  \sum_{i=1}^k\mathrm{D}_V^k\right)X + g\left( \sum_{i=1}^k\mathrm{D}_W^k\right)X
\end{align*}
C2:
\begin{align*}
\left( \sum_{i=1}^kf_k\mathrm{D}^k\right)_V(fX)&:= \sum_{i=1}^kf_k\mathrm{D}^k_V(fX)\\
&\overset{C2}{=}\sum_{i=1}^kf_k\left( \mathrm{D}^k_Vf\right)+\sum_{i=1}^kf_kf \mathrm{D}_V^kX\\
&=\sum_{i=1}^kf_k\left(Vf\right)X+\sum_{i=1}^kff_k \mathrm{D}_V^kX\\
&=\left( \left( \sum_{i=1}^kf_k\mathrm{D}^k\right)_Vf\right)X+f \left( \sum_{i=1}^kf_k\mathrm{D}^k\right)_VX 
\end{align*}
C3:
\begin{align*}
\left( \sum_{i=1}^kf_k\mathrm{D}^k\right) _V(X+Y)&:=\sum_{i=1}^kf_k\mathrm{D}^k_V(X+Y)\\
&\overset{C3}{=}\sum_{i=1}^kf_k\mathrm{D}^k_VX+\sum_{i=1}^kf_k\mathrm{D}^k_VY\\
&=\left( \sum_{i=1}^kf_k\mathrm{D}^k\right) _VX+\left( \sum_{i=1}^kf_k\mathrm{D}^k\right) _VY
\end{align*}
注意到$C2$中的$\mathrm{D}_Vf$按$Vf$理解,与联络无关.那么若$\sum_{i=1}^kf_k\not\equiv 1$,这时$(Vf)$部分是不相等,也就是说C2是不成立的,所以$2\mathrm{D}$不是联络.
\end{document}