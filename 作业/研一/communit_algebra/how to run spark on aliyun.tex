\section{如何在阿里云不同账号下建立内网通信}
1.选择相同地区的主机
2.开启专用网络VPC
3.开启高速通道,建立主机之间互连
4.设置路由表,指向欲连接的主机.这里有个问题,由于路由表里有指向交换机的项,如果192.168.1.x之间互连就会出问题.这时需要在
另一台机子上新建一个交换机182.168.2.0/24(254个可用ip),将实例私网ip接到这个交换机上改为192.168.2.x再去设置路由跳转
即可.这样3台机子两两之间建立高速通道即可实现内网互通,(不知道可不可以Slave1直接跳到Master连接Slave2).不同地域比如华东E和华东D之间交换器存在1ms延迟,但网速有2Gbps,比外网的1Mbps快太多.
\section{建立系统}
选择centos下的spark镜像降低配置成本
1.使用密码而不是密码对建立系统
2.如有需要重置密码
\section{配置SSH通信}
1.修改好Hostname与Hosts
接下来按教程配置好ssh通讯
2.记得删掉.ssh中已有的授权
\section{关于Hadoop的配置问题}
1.全部使用ip地址而不是Hostname
\section{Spark的配置}
每台机器的内存以及work数量,自行参照配置文件说明即可