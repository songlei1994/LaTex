\documentclass[b5paper]{ctexart}
\newcommand{\ts}[2]{#1\otimes #2} 
\newcommand{\tss}[3]{#1\otimes_{#2} #3} 
\newcommand{\es}[5]{$#1\xrightarrow{#2}#3\xrightarrow{#4}#5\xrightarrow{}0$} 
\newcommand{\ess}[5]{$0\xrightarrow{}#1\xrightarrow{#2}#3\xrightarrow{#4}#5\xrightarrow{}0$}
\RequirePackage{amsmath,amsthm,amsfonts,amssymb,bm,mathrsfs,wasysym}
\RequirePackage{fancyhdr}
\newsavebox{\mygraphic}
\sbox{\mygraphic}{\includegraphics[totalheight=1cm]{1.ps}}
\fancypagestyle{plain}{
\fancyhf{}
\fancyhead[LE]{\usebox{\mygraphic}\zihao{-4}~2017北大高代}
\fancyhead[LO]{\usebox{\mygraphic}\zihao{-4}2017北大高代}
\fancyhead[RO,RE]{博士数学论坛}
\fancyfoot[C]{\small -~\thepage~-}}
\RequirePackage[top=2cm,bottom=2cm,left=0.7cm,right=0.7cm]{geometry}
\renewcommand{\baselinestretch}{1.5}
\begin{document}
\pagestyle{plain}
\noindent
1.$(15')$
\[x_1=x_2=1,x_n=x_{n-1}+x_{n-2}.\]
试用矩阵论方法给出$x_n$通项.\\
2.$(15')$\\
$\alpha,\beta$为欧氏空间$V$中两个长度相等的向量.证明存在正交变换$A$使得$A\alpha=\beta$\\
3.$(10')$证明$n$阶$\mathrm{Hermite}$矩阵$A$有$n$个实特征值(考虑重数).\\
4.$(20')$.$F$为数域
\[\alpha_1,\alpha_2\cdots \alpha_n,\beta_1,\beta_2,\cdots \beta_n\]
是$F^n$中$2n$个列向量.
用\[\left|\alpha_1,\cdots \alpha_n\right|\]表示以$\alpha_1,\alpha_2\cdots \alpha_n$为列向量的矩阵的行列式.证明下面的行列式等式\\
\[\left|\alpha_1,\cdots \alpha_n\right|\cdot \left|\beta_1,\cdots \beta_n\right|=\sum_{i=1}^n \left|\alpha_1,\cdots \alpha_{i-1},\beta_1,\alpha_{i+1},\cdots \alpha_n\right|\cdot \left|\alpha_i, \beta_2,\cdots \beta_n\right|\]
5.$(20')$\\
$F$为数域,$V$是$F$上$n$维线性空间.$A$是$V$上线性变换.证明存在唯一可对角化线性变换$A_1$,幂零线性变换$A_2$
使得
\[A=A_1+A_2,A_1A_2=A_2A_1\]\\
6.$(20')$\\
$F$为数域,$A,B,P\in M_n(F)$,$P$幂零且\\
\[(A-B)P=P(A-B),BP-PB=2(A-B)\]
求一个可逆矩阵$Q$使得$AQ=QB$.
7.$(15')$.$\overrightarrow{a},\overrightarrow{b},\overrightarrow{c}$共面的充要条件为$\overrightarrow{a}\times \overrightarrow{b},
\overrightarrow{b}\times \overrightarrow{c},\overrightarrow{c}\times \overrightarrow{a}$共面
8.$(20')$空间中四点$O,A,B,C$使得
\[\angle AOB=\frac{\pi}{2},\angle BOC=\frac{\pi}{3},\angle COA=\frac{\pi}{4}\]
设$AOB$决定的平面为$\pi_1$,$BOC$决定的平面为$\pi_2$,求$\pi_1,\pi_2$二面角.求出二面角的余弦值即可.
9.$(15')$\\
$F$为单叶双曲面,$\overrightarrow{n}$为给定非零向量.
则空间中所有与$\overrightarrow{n}$垂直的平面与$F$交线的对称中心在一条直线上.
\end{document}