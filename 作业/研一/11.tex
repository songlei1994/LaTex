\documentclass[b5paper]{ctexart}
\newcommand{\ts}[2]{#1\otimes #2} 
\newcommand{\tss}[3]{#1\otimes_{#2} #3} 
\newcommand{\es}[5]{$#1\xrightarrow{#2}#3\xrightarrow{#4}#5\xrightarrow{}0$} 
\newcommand{\ess}[5]{$0\xrightarrow{}#1\xrightarrow{#2}#3\xrightarrow{#4}#5\xrightarrow{}0$}
\RequirePackage{amsmath,amsthm,amsfonts,amssymb,bm,mathrsfs,wasysym}
\RequirePackage{fancyhdr}
\fancypagestyle{plain}{
\fancyhf{}
\fancyfoot[C]{\small -~\thepage~-}}
\RequirePackage[top=2cm,bottom=2cm,left=0.7cm,right=0.7cm]{geometry}
\renewcommand{\baselinestretch}{1.5}
\usepackage{exscale} 
\usepackage{relsize}
\usepackage{fourier} 
\begin{document}
\pagestyle{plain}
\noindent
\zihao{-4}
\begin{center}
\textbf{\zihao{3}《怎样当一名科学家》读后感}
\end{center}
\hspace{250pt}数学学院~1601210073~宋雷
\par
第一眼看到书名为《如何当一名科学家》时,我是拒绝的.因为我是数学系在读研究生,大家都知道,一直以来就有人认为数学是``伪科学",因为它无法被证伪.我是自然不同意这种观点的.但我认为数学,特别是基础数学与其他实验科学(例如物理学,化学等一系列需要通过实验与现实世界联系的科学)有着很大的不同.在这篇读书报告,我更多的想简明的概述书中的观点,并努力将之迁移到我自己目前所处的学科,使之能够指导我将来的科学研究工作.\par
\textbf{第一章:科学的社会基础}彼得$\cdot$梅达很好的描绘了科学家性格各异的图景,对我触动很大.作为一个数学系研究生,我们天天和常人眼中``不自然"的数学概念打交道,以至于将这些概念视为``自然".再加之喜欢数学的人往往会性格上偏激,古怪(这是事实),这样便使我们难以用常人的眼光去看待世界.而是用逻辑的,冷冰冰的视角去看待问题,这样自然就少了几分温度.彼得的话提醒我们.尽管科学家们不是``常人",但他们也是人,也是具有非常不同性格,用不同方式做事的人.所以就算是从事基础数学研究,我们也应当充分重视自己作为人的那一部分特性,不要去排斥它.这是从个人的特异性来解读彼得的话.实际上作者引用彼得的话是想讨论科学家们这个整体的社会性.这是自然的,但对于数学工作者来说,大家普遍都不关心问题的社会层面,所以这里就不细细解读了.\par
\textbf{第二章:实验技术和数据处理}作者分析了实验中需要注意的问题,比如使用普遍接受的方法等等一列系的问题.但是与数学并无太大的关系.在数学中我们有绝对的对错评价,因为我们依据的是逻辑而不是``事实",结果的正确性并不会随其他因素而改变,所以这一章对我来说意义不大,故跳过.而\textbf{第三章,第五章}同理也跳过\par
\textbf{第四章:科学中的价值观}第一段写道:``科学家带到工作场所的不仅仅是一个技术工具箱.科学家必须作出关于解释数据,解决什么难题和何时结束实验的复杂决定."这点我是非常赞同的.在数学学习中,我们经常会遇到一些匪夷所思的定理与结论,一般的作者从来不会告诉你这个结果的来龙去脉.一般来说是作者自己也难以回想起理论建立的过程,抑或不愿花过多的时间去做这些``无用功".不写明思路过程是作者的权利,但是这往往会对其他人造成困扰,特别是那些证明过程繁杂的长篇累牍,如果作者不给出一个大体的框架,那对阅读者来说将是一个巨大的灾难.因此,我们不能将好的``数学成果"仅仅定义为能把大家吓一跳,让大家拍脑袋的结果.它应当是一个具有开创意义的,能被大多数同行业的人接受消化的理论.例如伽罗华的群论,虽然被埋没的几十年,但最终.同时,作者提到了哲学家对于科学价值判断的辅助作用,这点也适用于与哲学联系更为紧密的数学.\par
\textbf{第六章:发表和公开}很有用的一章.作者介绍了目前广泛使用的同行评议制度的历史渊源.并告诉我们:在学术杂志上发表的重要性还说明一种惯例,某项发现的大部分荣誉.倾向于由第一个发表这个观点或第一个发表这个发现的人.这就鼓励我们多多发表自己的新成果,以此来促进整个科学的发展\par
\textbf{第七章:荣誉分配,第八章:论文署名}在数学中,有一样东西是对于最终结果来说很重要,但它有不想最终的结果那样能实质化为定理或是推论.它就是想法.黎曼是以直代曲,无限分割求和求极限的结果.但其实在牛顿和莱布尼兹创立微积分时,根本就没有极限这个概念,他们只是隐隐约约,形式化的去演算那些``极限".黎曼积分最终命名为黎曼积分,但最重要的一部却在与``以直代曲"的想法.黎曼只是最后将之严格化,给了一个最终版本.同样在合作中,可能最后大家都忘了那个关键性的想法是由谁提出来的,而只记得最后的临门一脚是那个幸运而踢出.这是很不公平的.所以,我觉得对于科学中的荣誉问题,作为准科学家的我们应该有着更清晰的认识.并把这种认识落实到自己的行动当中.\par
还有我赞同论文署名按照贡献.\par
\textbf{第九章:科学中的错误与疏忽}虽然数学与实验科学都具有易错性,但数学的易错性并不是因为她的成果是暂时的,而是因为数学发展到今天,已经变为一门高度精神,极端复杂的一门学问,稍不留神就会犯意想不到的错误.作者提到了一个心理陷阱.如果发表的结果在之后证明存在由疏忽导致的错误,那么科学家会受到严厉的处理.这样会给研究者带来压力,而为了应对这中压力有时我们会采取牺牲质量的做法:发表事实上相同的研究结果,或者以``最小发表篇幅"发表其结果.无论哪种都会对整个科学研究环境带来不好的影响.为此,我们在做研究时,应该更加小心谨慎,避免把错误带入科学.否则,不管是对自己还是对科学,都是不负责任的表现,都有可能给双方带来不好的影响.\par
\textbf{第十章:科学中的不轨行为,第十一章:对违背道德标准的行为的反应}这章从社会层面介绍了不轨行为带来的恶劣影响.不轨行为的影响不仅仅维持在科学界诶.它的后果太严重,会伤害到科学领域以外的人.浪费公众的资金和给了那些批判科学的人实实在在的证据.同时,作者提到,对检举人的保护,应当是打击不轨行为时重要的一环,不然我们将会失去一个重要的信息来源.而第十一章,作者谈到许多揭发不道德行为时需要注意的事项.\par
总的来说,这本书提供了许多在科学研究中可能遇到的社会性问题的解答,有的是具体的案例,有的只是一个大概的思路,都具有很高的参考价值.特别是在文理分家的中国,我们这些理科男多多学习一些人文社科,与理科知识互补,我觉得是很有必要的,所以,最后衷心感谢这本书的作者和译者所作出的贡献!
\end{document}