\documentclass[a4paper]{ctexart}
\usepackage{amsmath}
\usepackage{amsthm}
\usepackage{amsfonts}
\usepackage{graphicx}
\usepackage{amssymb}
\usepackage{float}
\usepackage[all]{xy}
\begin{document}

\title{六月六日讨论班笔记}
\author{秘志桐}
\date{\today}
\maketitle
\renewcommand{\abstractname}{摘要}
\renewcommand{\refname}{参考文献}

\newtheorem{definition}{定义}
\newtheorem{theorem}{定理}
\newtheorem{lemma}{引理}
\newtheorem{corollary}{推论}
\newtheorem{prop}{命题}
\newtheorem{example}{例}
\newtheorem{remark}{注}
\begin{definition}贴$k$维胞腔:设$Y$为一拓扑空间,$e^k=\{x\in R^k: \|x\|<1\}$为$k$维胞腔。
记$k$维胞腔的边界为$S^{k-1}$,$0$维胞腔边界$S^{-1}$为空集。如果有连续映射$g:S^{k-1} \to Y$。则贴$k$维胞腔得到的贴空间$Y \cup_g e^k$定义如下
$$
Y \cup_g e^k=Y\amalg e^k/ \backsim
$$
其中,等价关系$\backsim$定义为$x\in S^{k-1} \backsim g(x)\in Y$。
\end{definition}

\begin{theorem}当$a$经过点$f(p),f(q),f(r),f(s)$时,描述$M^a=\{x \in M : f(x) \leqslant a\}$的同伦型的变化。
\begin{figure}[H]
\small
\centering
%\includegraphics[width=9cm]{2.jpg}\\
环面{}
\label{2}
\end{figure}
已经知道按照同胚的角度,我们有如下结果: \\
(1) $a<f(p)$,$M^a$为空集。\\
(2) $f(p)<a<f(q)$,$M^a$同胚于$2$维胞腔。\\
(3) $f(q)<a<f(r)$,$M^a$同胚于圆柱。\\
(4) $f(r)<a<f(s)$,$M^a$同胚于环面挖掉一个$2$维开胞腔。\\
(5) $f(s)<a$,$M^a$为环面。\\
下面从同伦的角度来观察上述变化。\\
(1)$\to$ (2)相当于向空集上贴一个$0$维胞腔。此时$M^a$为一$0$维胞腔,它同伦等价于$2$维胞腔。(这是因为$2$维胞腔可缩)\\
(2)$\to$ (3)相当于贴一个$1$维胞腔(下图左边)。此时$M^a$同伦等价于圆柱。
\begin{figure}[H]
\small
\centering
%\includegraphics[width=9cm]{761.jpg}\\
{}
\label{2}
\end{figure}
从圆柱到$M^a$的收缩映射由下式给出(示意图如下):    
\begin{figure}[H]
\small
\centering
%\includegraphics[width=9cm]{762.jpg}\\
{}
\label{2}
\end{figure}
收缩映射F定义如下
\begin{equation}
F(x,y,t)=
\begin{cases}
(tx,y) &\mbox{if $|y|\geqslant \frac{1}{2}$}\\
(t(x-\sqrt{{(\frac{1}{2}})^2-y^2})+\sqrt{{(\frac{1}{2}})^2-y^2},y) &\mbox{if $|y|\leqslant \frac{1}{2}$ }
\end{cases}
\end{equation}
上式是第一象限中收缩映射的坐标表示,利用对称性,可得到圆柱到$M^a$的收缩映射。\\
(3)$\to$ (4)相当于再贴一个$1$维胞腔。此时$M^a$同伦等于环面挖掉一个二维开胞腔。(见下图)
我们将证明上图左右两边都同伦等价于两个$S^1$的一点并。
对于右边,我们直接给出收缩映射,示意图如下(正方形挖掉一个圆收缩到四条边)。
收缩映射H定义如下
\begin{equation}
H(r,\theta,t)=
\begin{cases}
(t(1-rcos\theta)+rcos\theta,t(tan\theta-rsin\theta)+rsin\theta) &\mbox{if $0\leqslant \theta \leqslant \frac{\pi}{4}$}\\
(t(\frac{1}{tan\theta}-rcos\theta)+rcos\theta,t(1-rsin\theta)+rsin\theta) &\mbox{if $\frac{\pi}{4}\leqslant \theta < \frac{\pi}{2}$ }\\
(0,t(1-y)+y) &\mbox{if $\theta=\frac{\pi}{2}$}
\end{cases}
\end{equation}
这是第一象限中收缩映射的坐标表示,利用对称性,可得到环面挖掉一个二维开胞腔到两个$S^1$的一点并的收缩映射。\\
对于$M^a$(圆柱贴上一个$1$维胞腔),我们构造从如下图所示图形到正方形中两条直线$l_1,l_2$的形变收缩。为此,设$l_1$为$x$轴,$l_2$为$y$轴,正方形边长为$2$,两条直线交于原点O。图中红线所示菱形短边长为1,长边长为2,以原点为中心。
\begin{figure}[H]
\small
\centering
%\includegraphics[width=9cm]{765.jpg}\\
{}
\label{2}
\end{figure}
我们先将正方形沿横向收缩到菱形以及$l_2$上。收缩映射如下:
\begin{equation}
G(x,y,t)=
\begin{cases}
(((1-t)x,y) &\mbox{if $y \geqslant \frac{1}{2}$}\\
((1-t)x+t(1-2y),y) &\mbox{if $0\leqslant y\leqslant \frac{1}{2}$ }
\end{cases}
\end{equation}
这是第一象限中收缩映射的坐标表示,再利用对称性即可。
再构造收缩映射使得下图沿纵向收缩到$x,y$轴,从而得到两个$S^1$的一点并。
\begin{figure}[H]
\small
\centering
%\includegraphics[width=9cm]{766.jpg}\\
{}
\label{2}
\end{figure}
\begin{equation}
J(x,y,t)=
\begin{cases}
(x,(1-t)y) &\mbox{if $y \leqslant \frac{1}{2}$}\\
(0,(1+t)y-t) &\mbox{if $y\geqslant \frac{1}{2}$ }
\end{cases}
\end{equation}
综上可知,此时$M^a$与两个$S^1$的一点并同伦等价。这就证明了$M^a$与环面挖掉一个二维开胞腔同伦等价。\\
(4)$\to$ (5)相当于贴一个二维胞腔,此时$M^a$为一个环面\\

\end{theorem}



\begin{theorem}

点p,q,r,s为映射$f$的临界点,即切映射$df$在该点为零映射。
\begin{proof}
我们以p点为例,注意到环面的参数表示$\phi(\theta,\beta)$实际上给出环面在$p$点的局部坐标表示.要证明$df$在p点为零映射,只需证明$f$的局部坐标表示$\widetilde{f}$在p点对应的坐标处所有一阶偏导数均为0.注意到在局部坐标下:
$$
\widetilde{f}(\theta,\beta)=3-(2+cos\beta)cos\theta
$$
直接计算得
$$
\frac{\partial \widetilde{f}}{\partial \theta}=(2+cos\beta)sin\theta
$$
$$
\frac{\partial \widetilde{f}}{\partial \beta}=sin\beta cos\theta
$$
代入p点对应的坐标$(0,0)$知有
$\frac{\partial \widetilde{f}}{\partial \theta}(0,0)=0$与$\frac{\partial \widetilde{f}}{\partial \beta}(0,0)=0$。从而p点为$f$的临界点。
\end{proof}
\end{theorem}

\begin{remark}
事实上,由第二节的Morse Lemma知,在临界点附近,我们可以取适当的局部坐标,使得$f$的局部表示$\widetilde{f}$有形式$\widetilde{f}=\pm x^2+\pm y^2$。正负号的确定实际上与函数$f$在该临界点的“指数”有关。
\end{remark}


\begin{thebibliography}{9}
\bibitem{}
  Milnor,
   \emph{Morse Theory}.

\end{thebibliography}
\end{document}