\documentclass[b5paper]{ctexart}
\newcommand{\ts}[2]{#1\otimes #2} 
\newcommand{\tss}[3]{#1\otimes_{#2} #3} 
\newcommand{\es}[5]{$#1\xrightarrow{#2}#3\xrightarrow{#4}#5\xrightarrow{}0$} 
\newcommand{\ess}[5]{$0\xrightarrow{}#1\xrightarrow{#2}#3\xrightarrow{#4}#5\xrightarrow{}0$}
\RequirePackage{amsmath,amsthm,amsfonts,amssymb,bm,mathrsfs,wasysym}
\RequirePackage{fancyhdr}
\newsavebox{\mygraphic}
\sbox{\mygraphic}{\includegraphics[totalheight=1cm]{1.ps}}
\fancypagestyle{plain}{
\fancyhf{}
\fancyhead[LE]{\usebox{\mygraphic}}
\fancyhead[LO]{\usebox{\mygraphic}}
\fancyhead[RO,RE]{ \zihao{-4}宋雷~1601210073}
\fancyfoot[c]{\small -~\thepage~-}}
\RequirePackage[top=2cm,bottom=2cm,left=0.7cm,right=0.7cm]{geometry}
\renewcommand{\baselinestretch}{1.5}
\usepackage{exscale} 
\usepackage{relsize} 
\begin{document}
\pagestyle{plain}
\noindent
\zihao{-4}
1.设$\phi\simeq \phi',C_*\to C'_*$则$\phi\otimes id\simeq \phi'\otimes id,C_*\otimes D_*\to C'_*\otimes D_*$\\
验证即可,$C_*,C'_*$的边缘算子均记为$\partial_*$\\
由$\phi\simeq \phi'$知存在$T=\{T_q:C_q\to C'_{q+q}\}$足下式
\[\partial_{p+1}\circ T_p(c_p)+T_{p-1}\circ \partial_p(c_p)=\phi_p(c_p)-\phi'_p(c_p)\]
利用$T'_n(c_p\otimes d_q)=(T_pc_p)\otimes d_q$定义$(C_*\otimes D_*)_n\to (C'_*\otimes D_*)_{n+1}$的同态$T'$.那么
\[\begin{array}{l}
~~~~\partial^{\otimes}_{n+1}\circ T'_n(c_p\otimes d_q)+T'_{n-1}\circ \partial^{\otimes}_n(c_p\otimes d_q)\\
=\partial^{\otimes}_{n+1}(T_pc_p\otimes d_q)+T'_{n-1}(\partial_pc_p\otimes d_q+(-1)^p(c_p\otimes \delta d_q))\\
=(\partial_{p+1}\circ T_p)(c_p)\otimes d_q+(-1)^{p+1}(T_pc_q)\otimes (\delta_qd_q)+(T_{p-1}\circ \partial_p(C_p)\otimes d_q+(-1)^pT_pc_p\otimes \delta d_q\\
=[\partial_{p+1}\circ T_p(c_p)+T_{p-1}\circ \partial_p(c_p)]\otimes d_q\\
=[\phi_p(c_p)-\phi'_p(c_p)]\otimes [id(d_q)]\\
=(\phi\otimes id)_n(c_p\otimes d_q)-(\phi'\otimes id)_n(c_p\otimes d_q)
\end{array}\]
这就说明了$\phi\otimes id\simeq \phi'\otimes id$.\\
2.计算$K\times K$的$H^*,H_*$,其中$K$
为克莱因瓶\\
利用$K\"{u}nneth$公式,我们有
\[\begin{array}{c}
H_n(K\times K)\cong\bigoplus\limits_{p+q=n}H_p(K)\otimes
H_q(K)\bigoplus\bigoplus\limits_{p+q=n-1}Tor_1(H_p(K),H_q(K))\\
H^n(K\times K)\cong\bigoplus\limits_{p+q=n}H^p(K)\otimes
H^q(K)\bigoplus\bigoplus\limits_{p+q=n+1}Tor_1(H^p(K),H^q(K))\\
\end{array}\]
由此可以计算出.\\
\[H_q(K\times K)\cong\left\lbrace \begin{array}{ll}
\mathbb{Z} & q=0\\
\mathbb{Z}\oplus \mathbb{Z}\oplus \mathbb{Z}_2\oplus\mathbb{Z}_2 & q=1\\
\mathbb{Z}\oplus\mathbb{Z}_2\oplus\mathbb{Z}_2\oplus\mathbb{Z}_2 & q=2\\
\mathbb{Z}_2 & q=3\\
0 & else
\end{array}\right. \]
\[H^q(K\times K)\cong\left\lbrace \begin{array}{ll}
\mathbb{Z} & q=0\\
\mathbb{Z}\oplus\mathbb{Z} & q=1\\
\mathbb{Z}\oplus\mathbb{Z}_2\oplus\mathbb{Z}_2 & q=2\\
\mathbb{Z}_2\oplus\mathbb{Z}_2\oplus\mathbb{Z}_2 & q=3\\
\mathbb{Z}_2& q=4\\
0 & else
\end{array}\right. 
\]
3.设$f:X\to Y,\xi \in H^q(Y;R),u\in H_{p+q}(X;R)$,证$f_*(f^*\xi\frown u)=\xi\frown f_*u.$\\
先在链的水平上证明,由同调类的卡积定义可以自然地从链的卡积过渡到同调类的卡积.\\
$f_{\#}(f_{\#}\xi\frown u)=f^{\#}(\langle f^{\#}\xi,{}_qu\rangle u_p)=f_{\#}(\langle \xi,f_{\#}{}_qu\rangle u_p)=\langle \xi,f_{\#}{}_qu\rangle f_{\#}u_p=\langle \xi,{}_q(f_{\#}u)\rangle (f_{\#}u)_p=\xi\frown f_{\#}u$.\\
4.$\xi\in H^p(X;R),\eta\in\Huge^q(X;R),u\in H_{p+q}(X;R)$,证明,$\langle \xi\smile \eta,u\rangle=\langle \xi,\eta\frown u\rangle.$\\
同样先在链的水平上证明,由同调类的卡积上积定义可以自然地从链过渡到同调类
$\langle \xi\smile \eta,u\rangle=\langle\xi,{}_pu\rangle\langle\eta,u_q\rangle=\langle\xi,\langle \eta,{}_qu\rangle u_p\rangle=\langle\xi,\eta\frown u\rangle$
\end{document}