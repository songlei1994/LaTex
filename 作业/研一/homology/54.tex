\documentclass[b5paper]{ctexart}
\newcommand{\ts}[2]{#1\otimes #2} 
\newcommand{\tss}[3]{#1\otimes_{#2} #3} 
\newcommand{\es}[5]{$#1\xrightarrow{#2}#3\xrightarrow{#4}#5\xrightarrow{}0$} 
\newcommand{\ess}[5]{$0\xrightarrow{}#1\xrightarrow{#2}#3\xrightarrow{#4}#5\xrightarrow{}0$}
\RequirePackage{amsmath,amsthm,amsfonts,amssymb,bm,mathrsfs,wasysym}
\RequirePackage{fancyhdr}
\usepackage{exscale} 
\usepackage{relsize}
\usepackage{fourier} 
\usepackage{asymptote}
\usepackage{wrapfig}
\newsavebox{\mygraphic}
\sbox{\mygraphic}{\includegraphics[totalheight=1cm]{1.ps}}
\fancypagestyle{plain}{
\fancyhf{}
\fancyhead[LE]{\usebox{\mygraphic}}
\fancyhead[LO]{\usebox{\mygraphic}}
\fancyhead[RO,RE]{ \zihao{4}宋雷~1601210073}
\fancyfoot[C]{\small -~\thepage~-}}
\RequirePackage[top=2cm,bottom=2cm,left=0.7cm,right=0.7cm]{geometry}
\renewcommand{\baselinestretch}{1.5}
\begin{document}
\pagestyle{plain}
\noindent
\zihao{4}
1.证明:$f:S^n\to S^n$为偶映射,那么么$\deg f$为偶数.\\
我们介绍一个引理\\
\textbf{引理:}设$X$为可剖分空间,$f:S^n\to X$连续,并且满足$f(x)=f(-x),\forall x\in S^n$.那么$f_p(H_p(S^n))\subset 2H_p(X)$.\\
引理证明:由公理化的同调论我们知道实际上只存在一种同调论.我们利用单纯同调论对上述进行证明,不妨设$S^n=|\sum^n|,|X|=K$可以构造$f$的单纯逼近$\varphi:(\sum^n)^{(r)}\to K$,使得任意的$a\in ((\sum^n)^{(r)})^0,\varphi(a)=\varphi(-a),$则$f_{n}=\varphi_n$.不难验证$\varphi_n(Z_n((\sum^n)^{(r)}))\subset 2Z_n(K)$.\\
然后利用引理我们可以直接得到$\deg f$为偶数.\\
2.计算$Klein$瓶的整系数上同调环.\\
\begin{wrapfigure}{r}{0pt}
\begin{asy}
import math;
import geometry;
import graph;
size(200);
for(int i=0;i<3;++i)
{
draw((i,0)--(i+1,0),MidArrow);
}
for(int i=0;i<3;++i)
{
draw((i,3)--(i+1,3),MidArrow);
}
for(int i=0;i<3;++i)
{
draw((i,2)--(i+1,2),MidArrow);
}
for(int i=0;i<3;++i)
{
draw((i,1)--(i+1,1),MidArrow);
}
for(int i=0;i<3;++i)
{
draw((0,i+1)--(0,i),MidArrow);
}
for(int i=0;i<3;++i)
{
draw((1,i+1)--(1,i),MidArrow);
}
for(int i=0;i<3;++i)
{
draw((3,i)--(3,i+1),MidArrow);
}
for(int i=0;i<3;++i)
{
draw((2,i)--(2,i+1),MidArrow);
}
for(int i=3;i>1;--i)
{
draw((i-2,i-1)--(i-1,i),MidArrow);
}
for(int i=3;i>0;--i)
{
draw((i-1,i-1)--(i,i),MidArrow);
}
for(int i=3;i>1;--i)
{
draw((i-1,i-2)--(i,i-1),MidArrow);
}
draw((0,3)--(1,2),MidArrow);
draw((3,0)--(2,1),MidArrow);
label("$a$",(0,0),SW);
label("$b$",(1,0),S);
label("$c$",(2,0),S);
label("$a$",(3,0),S);
label("$a$",(0,3),SW);
label("$b$",(1,3),N);
label("$c$",(2,3),N);
label("$a$",(3,3),N);
label("$d$",(0,2),W);
label("$e$",(0,1),W);
label("$d$",(3,1),E);
label("$e$",(3,2),E);
label("$g$",(1,2),SE);
label("$h$",(2,2),SE);
label("$i$",(1,1),NW);
label("$k$",(2,1),NW);
\end{asy}
\end{wrapfigure}
如图给出了$Klein$瓶的一个单纯剖分.因为二维卡积一维,0维卡积任意维都是平凡的,我们只需计算一维的卡积即可.
$H_1(K)$的基为$\omega_1=[ab]+[bc]+[ca]$,$z_1=[ad]+[dc]+[ca]$.\\
而$H_2(K)$中任意一个二维单形在$H_2(K)$中不为0.
$H^1(K)$的基为$\omega^1=[bc]+[bk]+[ik]+[ih]+[gh]+[gc]+[bc]$.
在$[gih]$上$\langle\omega^1\smile \omega^1,[gih]\rangle=\langle\omega^1,[gi]\rangle\langle\omega^1,[ih]\rangle=0$,同理我们可以在其他二维单形上计算都为0.
所以$\omega^1\smile \omega^1=0$,也就是说$Klein$瓶整系数的上同调环完全是平凡的.
\end{document}