\documentclass[b5paper]{ctexart}
\newcommand{\ts}[2]{#1\otimes #2} 
\newcommand{\tss}[3]{#1\otimes_{#2} #3} 
\newcommand{\es}[5]{$#1\xrightarrow{#2}#3\xrightarrow{#4}#5\xrightarrow{}0$} 
\newcommand{\ess}[5]{$0\xrightarrow{}#1\xrightarrow{#2}#3\xrightarrow{#4}#5\xrightarrow{}0$}
\RequirePackage{amsmath,amsthm,amsfonts,amssymb,bm,mathrsfs,wasysym}
\RequirePackage{fancyhdr}
\newsavebox{\mygraphic}
\sbox{\mygraphic}{\includegraphics[totalheight=1cm]{1.ps}}
\fancypagestyle{plain}{
\fancyhf{}
\fancyhead[LE]{\usebox{\mygraphic}}
\fancyhead[LO]{\usebox{\mygraphic}}
\fancyhead[RO,RE]{ 宋雷~1601210073}
\fancyfoot[C]{\small -~\thepage~-}}
\RequirePackage[top=2cm,bottom=2cm,left=0.7cm,right=0.7cm]{geometry}
\renewcommand{\baselinestretch}{1.5}
\begin{document}
\pagestyle{plain}
\noindent
2.不难看出在集合水平上,若$h\in H$,那么$hH=Hh=H$.如果$HaK\cap HbK\neq\emptyset,$则存在$h_1,h_2\in H,k_1,k_2\in K$使得$h_1ak_1=h_2bk_2$,由此式我们有$HaK=(Hh_1)a(k_1K)=H(h_1ak_1)K=H(h_2bk_2)K=(Hh_2)b(k_2K)=HbK$.\\
5.首先我们有$(g_1g_2)^{n_1n_2}=g_1^{n_1n_2}g_2^{n_1n_2}=(g_1^{n_1})^{n_2}(g_2^{n_2})^{n_1}=e$;记$o(g_1g_2)=t,$如果$t<n_1n_2$,那么存在$s<t,m\in\mathbb{Z}$使得$n_1n_2=mt+s$,于是$e=(g_1g_2)^{n_1n_2}=(g_1g_2)^{mt+s}=[(g_1g_2)^t]^m(g_1g_2)^s\Rightarrow (g_1g_2)^s=e$,这与$o(g_1g_2)=t$矛盾,于是$n_1n_2=t=o(g_1g_2).$\\
在$S_3$中取
\[g_1=\left( \begin{array}{lll}
1 & 2 & 3\\
1 & 3 & 2
\end{array}\right) ,
g_2=\left( \begin{array}{lll}
1 & 2 & 3\\
2 & 3 & 1
\end{array}\right) .
\]
那么
\[g_1g_2=\left( \begin{array}{lll}
1 & 2 & 3\\
2 & 1 & 3
\end{array}\right) ,
g_2g_1=\left( \begin{array}{lll}
1 & 2 & 3\\
3 & 2 & 1
\end{array}\right) ,
\]
于是$g_1g_2\neq g_2g_2$,并且$o(g_1)=2,o(g_2)=3,o(g_1g_2)=2\neq 6$.\\
8.若$ab\in AB$并且$ab\in C$,我们由$a\in C
$得到$b=a^{-1}(ab)\in C$,故$b\in B\cap C$,则$ab\in A(B\cap C)$.
对$ab\in A(B\cap C)$,由$b\in B\cap C$可知$ab\in AB$并且$ab\in C$,那么$ab\in AB\cap C$,
综上我们有$AB\cap C=A(B\cap C).$\\
13.首先我们取自同构为$g:a\rightarrow g^{-1}ag$,由题设可知$g^{-1}ag=a\Rightarrow ga=ag$,由$g,a$的任意性可知$G$为交换群.再取$\alpha:g\rightarrow g^{-1}$,由EX.12可知这是交换群的自同构,那么$g^{-1}=g$
.利用交换群的结构定理及$g^{-1}=g$,我们可以确定$\mathbb{Z},\mathbb{Z}_{p^m},\mathbb{Z}_{2^n},(p\geq 3,n\geq 2)$不在直和因子内.于是$G=\bigoplus\limits_m \mathbb{Z}_2$,我们先给出$\mathbb{Z}_2\oplus \mathbb{Z}_2$的一个非平凡自同构$\varphi$
\[
\begin{array}{ll}
(0,0)\rightarrow (0,0) \quad ,& (0,1)\rightarrow (1,0)\\
(1,0)\rightarrow (0,1) \quad ,& (1,1)\rightarrow (1,1)
\end{array}\]
若$m\geq 2$,那么$\varphi\oplus id$给出了$G$的一个非平凡自同构,这与题设矛盾.于是$m\leq 1$,从而$G=1,\mathbb{Z}_2.$
\end{document}