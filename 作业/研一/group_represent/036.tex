\documentclass[b5paper]{ctexart}
\newcommand{\ts}[2]{#1\otimes #2} 
\newcommand{\tss}[3]{#1\otimes_{#2} #3} 
\newcommand{\es}[5]{$#1\xrightarrow{#2}#3\xrightarrow{#4}#5\xrightarrow{}0$} 
\newcommand{\ess}[5]{$0\xrightarrow{}#1\xrightarrow{#2}#3\xrightarrow{#4}#5\xrightarrow{}0$}
\RequirePackage{amsmath,amsthm,amsfonts,amssymb,bm,mathrsfs,wasysym}
\RequirePackage{fancyhdr}
\usepackage{tikz}
\usepackage{wrapfig}
\newsavebox{\mygraphic}
\sbox{\mygraphic}{\includegraphics[totalheight=1cm]{1.ps}}
\fancypagestyle{plain}{
\fancyhf{}
\fancyhead[LE]{\usebox{\mygraphic}\zihao{-4}~群表示论~\today}
\fancyhead[LO]{\usebox{\mygraphic}\zihao{-4}~群表示论~\today}
\fancyhead[RO,RE]{\zihao{-4} 宋雷~1601210073}
\fancyfoot[C]{\small -~\thepage~-}}
\RequirePackage[top=2cm,bottom=2cm,headsep=0.5cm,footskip=1cm,left=1.0cm,right=1.0cm]{geometry}
\renewcommand{\baselinestretch}{1.5}
\begin{document}
\pagestyle{plain}
\zihao{-4}
\noindent
\\
1.将$v$扩充为$V_n(F)$的一组基$v,v_2,\cdots,v_n$,由于$g$在$V_n(F)/Fv$上诱导出恒等映射,那么$gv_i=v_i+l_iv,l_i\in F,i=2,\cdots,n$.不妨设$l_2\neq 0$,以此构造另一组基$v,v_2,\cdots,v_n-l_2^{-1}l_nv$,那么$g(v_i-l_2^{-1}l_iv_2)=v_i+l_iv-l_2^{-1}v_2-l_2(l_2^{-1}l_i)v=v_i-l_2^{-1}l_iv,i=3,\cdots,n$.于是在这组基下$g$具有$X_{12}(\alpha)$的形式,从而它们是共轭的.\\
2.(a)我们对维数使用数学归纳法进行证明.$n=1,2$时显然成立.对于完全旗\\
$\{V_1,\cdots,V_n\},\{W_1,\cdots,W_n\}$,如果$V_1=W_1$,那么对这两个完全旗做关于$V_1$的商空间,得到$V_{n-1}(F)=V_{n}(F)/V_1$上的完全旗.由归纳假设存在一组$B$-基,这组商集的$B$基的代表元和$V_1$可以组成一组基.同时,$*$的存在性是由于$\overline
{b_t}\in\overline{V_i},\overline{b_t}\notin\overline{V_{i-1}}$可知$b_t+Fv\in V_i,b_t+Fv\notin V_{i-1}$.若不唯一,则与归纳假设矛盾.\\
至于$V_1\neq W_1$,则取$w,v$分别为$V_1,W_1$的基.做商集$(V_i+Fw)/(Fv+Fw),(W_i+Fv)/(Fv+Fw)$,这会变成$V_{n-2}(F)=V_n(F)/(Fv+Fw)$的完全旗.以旗$V$为例,$\overline{V_i}=\overline{V_{i-1}}\Leftrightarrow V_i=V_{i-1}+Fw$.再由$w\notin V_1,w\in V_n$可知,一定存在唯一一个$i$,使得等号成立,此时$w\in V_i,w\notin V_{i-1}$,加之$(V_1+Fw)/(Fw+Fv)=0$,这就说明旗子长度变为$n-2
$.在这两个完全旗上使用归纳假设,得到一组$B$基.我们需要对它进行些许修正,设$\overline{\alpha}$是满足$\overline{\alpha}\in\overline{V_i},\overline{\alpha}\notin\overline{V_{i-1}},\overline{\alpha}\in\overline{W_j},\overline{\alpha}\notin\overline{W_{j-1}}$,那么$\alpha\notin W_{j-1},V_{j-1}$; $\alpha+Fw+Fv\in V_i+Fw$,也就是说,存在$f_w\in F$,使得$\alpha+f_ww\in V_i$,同理存在$f_v\in F$使得$\alpha+f_vv\in W_j$,用$\alpha+f_vv+f_ww$代替$\alpha$,加上$v,w$可以得到要求的$B$基,注意到对$\alpha$进行修正时并未改变其所属陪集.同上唯一性与存在性是显然的.\\
(b)对于标准完全旗$\{V_1,\cdots,V_n\}$,$\forall g\in G$找到这样一组$B$基,由$b\in V_i,b\notin V_{i-1}$可知,$b=\sum_{j=1}^ia_jv_i,a_i\neq 0$,那么完全旗的一组基可在$B$的变换下变为$B$基(顺序一般不完全对应).从而由所述题意$\omega b_1(v_1,v_2,\cdots,v_n)= b_2(gv_1,\cdots,gv_n)=b_2g(v_1,\cdots,v_n)$从而$g=b^{-1}_2\omega b_1$,所以$G\subset BWB$.\\
4.$P= U_P\rtimes L_P$是显然的.关键在于证明$P=N_G(U_P)$,只需证明$N_G(U_P)\subset P$即可.使用数学归纳即可.我们用类似的符号去记分块矩阵.任取$B=(b_{ij})\in N_G(U_P)$,利用$BX_{in}(\alpha)=U_PB$,可知$b_{ni}\alpha=0$,注意到$\alpha$为任意矩阵,通过取$\alpha$的第一列为标准基$v_i$可知,$b_{ni}=0,i=1,\cdots,n-1$.这时$B$矩阵具有如下性质
\[\left( \begin{array}{cc}
B' & A\\
0 & C
\end{array}\right)
\left( \begin{array}{cc}
U' & \ast\\
0 & \ast
\end{array}\right)=\left( \begin{array}{cc}
U'' & \ast\\
0 & \ast
\end{array}\right)
\left( \begin{array}{cc}
B' & A\\
0 & C
\end{array}\right) \]
同时注意到
\[\left( \begin{array}{cc}
B' & A\\
0 & C
\end{array}\right)^{-1}=\left( \begin{array}{cc}
B'^{-1} & -B'^{-1}AC^{-1}\\
0 & C^{-1}
\end{array}\right)\]
于是$B'U'B'^{-1}$也是分块矩阵.这里我们利用归纳假设,不难看出$B'$是分块上三角可逆矩阵,且与$U'$有相同的分块.从而$B$为与$U$具有相同分块的分块上三角矩阵,即$N_G(U_P)\subset P$.\\
1.注意到$U\subset B\cap SL(n,F)\subset H$,那么$B=\langle U,T\rangle\subset\langle H,T\rangle$,由定理可知$H'=\langle H,T\rangle$为可逆分块上三角矩阵群,那么$HT=TH$.我们将证明$H=H'\cap SL(n,F)$,只需证明$H'\cap SL(n,F)\subset H$即可.任取$h\in H'\cap SL(n,F)$我们有$h=h_1t_1,h_1\in H,t_1\in T
$,而$\det(t_1)=\det(h)/det(h_1)=1$,注意到$t_1\in T\cap SL(n,F)\subset B\cap SL(n,F)\subset H$,那么$H'\cap SL(n,F)\subset H$.从而$H$为分块对角矩阵,同时满足$\det(H)=1.$证明过程实际上给了一个$GL(n,F)$中包含$B$的子群与$SL(n,F)$中包含$B\cap SL(n,F)$的子群的一个对应,即映射$\langle \cdot,T\rangle$,$\cdot\cap SL(n,F)$.\\
2.$PSL(2,2)\cong SL(2,2)\cong \Sigma_3.$\\
对于$PSL(2,3)\cong A_4$,只需要证明存在非正规的Sylow~3子群即可.\\
通过计算可知$PSL(2,3)$中存在4个Sylow~3子群,它们的生成元如下
\[
\overline{\left( \begin{array}{cc}
1 & -1\\
1 & 0
\end{array}\right) }\quad\overline{\left( \begin{array}{cc}
1 & 1\\
-1 & 0
\end{array}\right) }\quad
\overline{\left( \begin{array}{cc}
1 & 0\\
1 & 1
\end{array}\right) }\quad
\overline{\left( \begin{array}{cc}
1 & 1\\
0 & 1
\end{array}\right) }\]
从而存在非正规Sylow~3子群$H$.那么关于$H$的传递置换的核$H_G=1$,于是$G$在$H$上的置换表示是忠实的.由于$|PSL(2,3):H|=4$,那么$PSL(2,3)\lesssim S_4$,而$S_4$中只有一个$12$阶子群$A_4$,于是$PSL(2,3)\cong A_4$.\\
3.我们只需要证明$60$阶单群同构于$A_5$即可.\\
考虑$60$阶单群的传递置换表示.由于$G$是单群,每个置换表示都是忠实的,因此$G$不能有到$S_n(n\leq 4)$中的置换表示,这说明$G$中不存在指数$\leq 4$的子群.\\
下面我们证明$G$有12阶子群.根据Sylow第三定理,$G$中Sylow~2子群的个数$n_2=3,5,15$.因为$n_2$是Sylow~2子群的正规化子的指数,前面已证$G$中无指数为3的子群,故$n_2=5,15$.若$n_2=5$,则$G$中已有指数为5的子群.若$n_2=15$,又假定$G$的两个Sylow~2子群的交均为1,则$G$中2元素共有$1+3\times 5=46$,而由Sylow定理,$G$的Sylow~5子群个数$n_5=6$,非单位的5元素个数为$4\times 6=24.$于是2元素与5元素总数已超过群的阶,故不可能.这说明必有$G$的两个Sylow~2子群之交为2阶群$A$.考虑$A$的中心化子$C_G(A)$,它已包含两个$Sylow~2$子群,故其阶$>4$,并且是4的倍数.前面已证$G$中无指数$\leq 4$的子群,于是$|C_G(A)|=12$.\\
那么$G$在这个12阶子群上的传递置换把$G$嵌入$S_5$,而$S_5$中只有一个60阶子群$A_5$.所以$G\cong A_5$.
\end{document}