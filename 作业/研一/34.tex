\documentclass[b5paper]{ctexart}
\newcommand{\ts}[2]{#1\otimes #2} 
\newcommand{\tss}[3]{#1\otimes_{#2} #3} 
\newcommand{\es}[5]{$#1\xrightarrow{#2}#3\xrightarrow{#4}#5\xrightarrow{}0$} 
\newcommand{\ess}[5]{$0\xrightarrow{}#1\xrightarrow{#2}#3\xrightarrow{#4}#5\xrightarrow{}0$}
\RequirePackage{amsmath,amsthm,amsfonts,amssymb,bm,mathrsfs,wasysym}
\RequirePackage{fancyhdr}
\newsavebox{\mygraphic}
\sbox{\mygraphic}{\includegraphics[totalheight=1cm]{1.ps}}
\fancypagestyle{plain}{
\fancyhf{}
\fancyhead[LE]{\usebox{\mygraphic}\zihao{-4}~2016北大高代}
\fancyhead[LO]{\usebox{\mygraphic}\zihao{-4}~2016北大高代}
\fancyhead[RO,RE]{博士数学论坛}
\fancyfoot[C]{\small -~\thepage~-}}
\RequirePackage[top=2cm,bottom=2cm,left=0.7cm,right=0.7cm]{geometry}
\renewcommand{\baselinestretch}{1.5}
\begin{document}
\pagestyle{plain}
\noindent
1.$(10')$在$R^3$上定义线性变换$A,A$在自然基
\[\varepsilon_1=\left(\begin{array}{c}
1\\
0\\
0\end{array}\right),\varepsilon_2=\left(\begin{array}{c}
0\\
1\\
0\end{array}\right),\varepsilon_3=\left(\begin{array}{c}
0\\
0\\
1\end{array}\right)\]下的矩阵为
\[\left(\begin{array}{ccc}
0& 1& -1\\
0& 0& 1\\
0& 0& 0\end{array}\right)\]
求$R^3$的一组基,使得$A$在这组基下具有Jordan型\\
2.$(10')$3阶实矩阵$A$的特征多项式为$x^3-3x^2+4x-2$证明$A$不是对称阵也不是正交阵.\\
3.$(15')$在所有2阶实方阵上定义二次型$f$:$X\rightarrow Tr(X^2)$求$f$的秩和符号差.\\
4.$(15')$设$V$是有限维线性空间,$A,B$是$V$上线性变换满足下面条件\\
(1)$AB=O$.这里$O$是0变换\\
(2)$A$的任意不变子空间也是$B$的不变子空间\\
(3)$A^5+A^4+A^3+A^2+A=O$\\
证明$BA=O$\\
5.$(15')$设$V$是全体次数不超过$n$的实系数多项式组成的线性空间定义线性变换$A$:$f(x)\rightarrow f(1-x)$,求$A$的特征值和对应的特征子空间.\\
6.$(15')$计算行列式.各行底数为等差数列,各列底数也为等差数列,所有指数都是$50$
\[\left|\begin{array}{ccccc}
1^{50}& 2^{50}& 3^{50}& \cdots & 100^{50}\\
2^{50}& 3^{50}& 4^{50}& \cdots & 101^{50}\\
\vdots& \vdots&\vdots&\vdots& \vdots\\
100^{50}& 101^{50}& 102^{50}& \cdots& 199^{50}\\\end{array}\right|\]
7.$(20')$设$V$是复数域上有限维线性空间$A$是$V$上线性变换,$A$在一组基下矩阵为$F$\\
(1)若$A$可对角化对任意$A$的不变子空间$U$,存在$U$的一个补空间$W$是$A$的不变子空间;\\
(2)若对任意$A$的不变子空间$U$,存在$U$的一个补空间$W$是$A$的不变子空间,证明$F$可对角化.\\
8.$(20')$平面上一个可逆仿射变换将一个圆映为椭圆或圆.详细论证这一点\\
9.$(15')$平面$Ax+By+Cz+D=0$与双曲抛物面$2z=x^2-y^2$交于两条直线
证明$A^2-B^2-2CD=0$\\
10.$(15')$正十二面体有12个面,每个面为正五边形,每个顶点连接3条棱.求它的内切球与外接球半径比.\\
\end{document}