% !TeX encoding = UTF-8
\documentclass[b5paper]{ctexart}
\newcommand{\qa}[2]{\left[ \substack{#1\vspace{6pt}\\#2}\right]}
\RequirePackage{amsmath,amsthm,amsfonts,amssymb,bm,mathrsfs,wasysym}
\RequirePackage{fancyhdr}
\usepackage{tikz}
\usepackage{wrapfig}
\newsavebox{\mygraphic}
\sbox{\mygraphic}{\includegraphics[totalheight=1cm]{1.ps}}
\fancypagestyle{plain}{
\fancyhf{}
\fancyhead[LE]{\usebox{\mygraphic}\zihao{-4}~组合数学~\today}
\fancyhead[LO]{\usebox{\mygraphic}\zihao{-4}~组合数学~\today}
\fancyhead[RO,RE]{\zihao{-4} 宋雷~1601210073}
\fancyfoot[C]{\small -~\thepage~-}}
\RequirePackage[top=2cm,bottom=2cm,headsep=0.5cm,footskip=1cm,left=1.0cm,right=1.0cm]{geometry}
\renewcommand{\baselinestretch}{1.5}
\begin{document}
\pagestyle{plain}
\zihao{-4}
\noindent
\\
4.定义$h(\omega_j)$为第$j$步向右走后,横线所在的高度,这个量与$\omega_j$之前1的个数,以及$\omega_j$代表的横线与横轴所夹的方格数相同.我们固定住$j$,去求$i<j$时$\omega_i<\omega_j$的个数
\[coinv(\omega)=\sum_{j=0}^n\sum_{\substack{i<j\\
\omega_i<\omega_j}}1=\sum_{\omega_j=1}\sum_{\substack{i<j\\
\omega_i<\omega_j}}1=\sum_{\omega_j=1}h(\omega_j)=area(CW_n)+\dbinom{n+1}{2}\]
那么
\[\sum_{\omega\in CW_n}q^{coinv(\omega)-\binom{n+1}{2}}=\sum_{\omega\in CW_n}q^{area(\omega)}=C_n(q)\]
7.(a)利用数学归纳法,去计算$x^k$的系数
\[q^{\binom{k}{2}}\qa{n}{k} +q^nq^{\binom{k-1}{2}}\qa{n}{k-1}=q^{\binom{n}{2}}\qa{n}{k} +q^{n+1-k}\qa{n}{k-1}=q^{\binom{k}{2}}\qa{n+1}{k}\]
(b)\\
9.(a)$s(n,1)$为$S_n$中轮换数为1的置换个数.这相当于$[n]$的圆排列数,为$(n-1)!$,加上符号即为$(-1)^{n-1}(n-1)!$.\\
(b)$S_n$中恰有$n-1$个轮换,则完全由那个$2-$轮换决定,个数为$\dbinom{n}{2}$.\\
10.(a)将$[n]$分为1块的划分数自然是1.\\
(b)利用递推关系$S(n,2)=2S(n-1,2)+S(n-1,1)$即可证明.\\
(c)分成$n-1$块时,分法完全由那个2元子集决定,故为$\dbinom{n}{2}$.\\
(d)按单元子集的个数分类,不难得知单子集个数只能是$n-3$,$n-4$.有$n-2$个单元子集的分法数为$\dbinom{n}{3}$,有$n-4$个单元子集,还需要2个分块,但每个分块要大于1,故只能都是2元分块,而4个元素分为2块有4种可能,合在一起即为$\dbinom{n}{3}+4\dbinom{n}{4}$\\
11.(b)记$h(x)=fg$,$\deg{f(x)}=n,\deg{g(x)}=m$.那么$x^{m+n}h(x^{-1})=[x^nf(x^{-1})][x^mg(x^{-1})]=f(x)g(x)=h(x)$,由充要条件可知$h(x)$是自反的.
\end{document}