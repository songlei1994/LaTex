% !TeX encoding = UTF-8
\documentclass[b5paper]{ctexart}
\newcommand{\qa}[2]{\left[ \substack{\vspace{1pt}\\#1\vspace{3pt}\\#2\vspace{1pt}}\right]}
\RequirePackage{amsmath,amsthm,amsfonts,amssymb,bm,mathrsfs,wasysym}
\RequirePackage{fancyhdr}
\usepackage{tikz}
\usepackage{wrapfig}
\newsavebox{\mygraphic}
\sbox{\mygraphic}{\includegraphics[totalheight=1cm]{1.ps}}
\fancypagestyle{plain}{
\fancyhf{}
\fancyhead[LE]{\usebox{\mygraphic}\zihao{-4}~组合数学~\today}
\fancyhead[LO]{\usebox{\mygraphic}\zihao{-4}~组合数学~\today}
\fancyhead[RO,RE]{\zihao{-4} 宋雷~1601210073}
\fancyfoot[C]{\small -~\thepage~-}}
\RequirePackage[top=2cm,bottom=2cm,headsep=0.5cm,footskip=1cm,left=1.0cm,right=1.0cm]{geometry}
\renewcommand{\baselinestretch}{1.5}
\begin{document}
\pagestyle{plain}
\zihao{-4}
\noindent
\\
\textbf{\zihao{4}补交第6次作业}\\
12.(a)对应$x_1+x_2+\cdots+x_m=n$的非负解个数为$\binom{n-1}{m-1}$.\\
(b)$(x+x^2+\cdots+x^k)^m$\\
(c)令$a_{m+1}=a_0$,则$x+x^2+\cdots+x^m$是单峰且自反的.由11知$\{c(k,m,n)\}$的生成函数是单峰且自反的,那么自然有$(c)(d)$成立.\\
13.\[\lim_{q\to 1}\left[ r\right]=\lim_{q\to 1}\dfrac{1-q^r}{1-q}=r\]
(2)$\qa{r}{k+1}=\dfrac{(1-q^{r-k})\cdots(1-q^r)}{(1-q)\cdots(1-q^{k+1})}=\dfrac{1-q^{r-k}}{1-q^{k+1}}\qa{r}{k}$,两边取极限并使用归纳假设即可.\\
2.(a)如果这条闭途径不是奇圈,设奇圈为$v_0v_1\cdots v_{2n+1}v_0$,那么一定有某个$v_i$出现了两次即出现形式$v_0line(1)v_iline(2)v_iline(3)v_0$,那么$v_0line(1)v_iline(3)v_0,v_iline(2)v_i$恰好将原来的奇途径分为两段,那么至少有一个为奇途径,从而由归纳法原\\
(b)如途径$v_0v_1v_2v_3v_2v_4v_0$中就没有.\\
3.若$K_3\not\subseteq G$,则对任意$uv\in E$,有$d(u)+d(v)\leq 2n$,从而
\[\sum_{uv\in E}(d(u)+d(v))\leq 2n(n^2+1)\]
另一方面,有
\[\sum_{uv\in E}(d(u)+d(v))=\sum_{u\in V}d^2(u)\geq \dfrac{\left( \sum_{u\in V}d(u)\right)^2}{2n},\]
从而$n^2+1\leq n^2$,矛盾.\\
5.考虑图中一条最长路$v_1v_2\cdots v_n$,由于$\delta(G)\geq 3$,那么$d(v_1)\geq 3$,由于$v_1$与路之外的顶点有连接都会增加路的长度,与最大长度矛盾,于是$v_1$只能和路中其他的顶点向连接,设为$v_i,v_j$,对于圈$v_1v_2\cdots v_iv_1,v_1v_2\cdots v_jv_1,v_1v_jv_{j-1}\cdots v_iv_1$,不难计算出它们的边之和为偶数,那么这三个圈中,至少有一个为偶圈.\\
7.从与最大的度顶点$u$出发的$\Delta(T)$个顶点,各找一条不通过$u$与其他顶点的最长路,那么路径的终点一定是一片叶子,而且不同的顶点出发路径顶点一定不交,不然会有圈.\\
\end{document}