\documentclass[a4paper]{ctexart}
\usepackage{amsmath,amsthm,amsfonts,amssymb,bm,mathrsfs,wasysym}
\usepackage[top=2cm,bottom=2cm,left=1cm,right=1cm]{geometry}
\renewcommand{\baselinestretch}{1.5}
\usepackage{exscale} 
\usepackage{relsize}
%\usepackage{fourier} 
\begin{document}
\noindent
\textbf{辅导科目}:数学分析,高等代数,解析几何\\
\textbf{授课课时}:30课时(一个课时45分钟)\\
\textbf{授课老师}:宋雷\\
\textbf{授课教材}:数学分析(一)(二)(三) 伍胜建\\
高等代数学习指导书(上下) 丘维声\\
解析几何 尤承业\\
\zihao{4}\textbf{课程安排}:\\
\zihao{-4}
\textbf{夯实基础阶段}:分为课上与课后两个时间段.课后学生自主阅读数学分析(一)(二)(三),高等代数学习指导书(上下),重点熟悉老师指出的重难点,完成老师布置的相应的课后习题.而课上老师主要以答疑为主,穿插着讲一些系统性的总结与感悟.并在上课的过程中了解学生,为接下来的阶段打下基础.\\
\textbf{拔高能力阶段}:根据学生第一阶段的复习情况以及学生自身学习的特点,选择合适的习题集与辅导书,进行专门得特训.加深学生对北京大学基础数学考研的难度以及考察要求的理解,建立完整的知识体系,学会一些典型题目的典型解法.\\
\textbf{真题研究阶段}:任务有二:一是分享老师本人对于北大考研的一些经验与看法,包括老师对于历年考试题目的一个点评,对于今年考题的一个预测.二是训练学生对于解北大数学题的能力,训练方法包括但不限于,真题的深入剖析,相关习题的实战演练.通过这两个任务,希望能够使学生了解北京大学数学系的命题风格,考察要点,并能熟练掌握相应的应对方法.\\
每个阶段的课时按照学生的进度进行动态调整.
\end{document}