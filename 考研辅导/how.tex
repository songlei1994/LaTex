\documentclass[a4paper]{ctexart}
\usepackage{amsmath,amsthm,amsfonts,amssymb,bm,mathrsfs,wasysym}
\usepackage[top=2cm,bottom=2cm,left=1cm,right=1cm]{geometry}
\renewcommand{\baselinestretch}{1.5}
\usepackage{exscale} 
\usepackage{relsize}
%\usepackage{fourier} 
\begin{document}
\noindent
\subsection{自身的基础}
自己制定一个详细的学习作战计划就差不多了.报班能给你指一条明路,让你复习的时候能少走弯路,帮助你对知识点理解的加深,以弥补自身薄弱的基础.

\subsection{自己的学习能力}

如果你自己学习能力很强,自己平时能总结出很多的各种学习方法,那你学习成绩肯定很好,那么考研那点东西对你来说就不是什么大问题了,虽然考研试题很难,假设你的学习能力一般,难一点的东西你需要看几遍才能看懂,那建议你自己报个班,老师跟你将几遍,再总结一些方法与技巧,你能省去很多弯路.

\subsection{自觉性}

如果你自己自觉性很好,两耳不闻窗外事,一心只读圣贤书,那你就看看学习能力,然后再决定报不报班,一般这种人是不需要报班的;但是如果你自觉性比较差,闻鸡不起舞,夜战LOL,自习玩手机,那你就报个吧,能监督你学习,这样能让你收收心,把心思放在学习上。


1.现在开始报班能让你早早的进入复习状态,不至于想考研想复习但是没有实际行动。

2.报班老师给你讲一遍能加深你对考试试题的理解,能学到各种应试技巧,答题技巧,毕竟考研是应试考试,所以这个还是很重要的。

3.报班老师能让你准确的抓住考试的重点、难点,能看透大纲的新增点,删除点等等,毕竟那些老师在考研这块领域是咱们普通学生的爷爷辈的,经验比咱们丰富多了

4.考研机构的名师很多都是身怀绝技,幽默风趣,让你在考研路上能不少那么的无聊,相当于调节自己了。

\section{师资力量}

师资力量是考察辅导班的首要因素,考生可以针对辅导名师的辅导年限、辅导经验、历年辅导效果和学员评价等因素进行综合评价,询问往届学长然后选择或者在网上看大家对该老师的评价。

判断师资力量关键在于综合实力,因为任何一门课程,都不是由一、两个教师包到底的,而是一批教师互相配合的结果。还要深入了解教师的学术背景、资料著述成就和辅导成就等。

\subsection{课程及资料}

考虑课程的时间是不是充足,是否有配套的资料或者讲义,这是需要重点考虑的地方之一。

\subsection{个人的感觉}

最后这个的意思是你要报班一定要找那种你喜欢的风格的老师,你喜欢幽默风趣的还是喜欢成熟稳重的,喜欢说话如华少的,还是喜欢慢慢说话给你反应时间的,这些都是你需要了解然后再决定的。

1、看看学长学姐是如何复习专业课、公共课的,他山之石,可以功玉,与其闭门造车,不如看看前人经验。其他机构提供不了,凯程可以提供大量学员的经验谈视频。

2、深入研究各个专业考研的专业特点,做到精准制导,同时深入发掘每个学生独特的潜质,呈现出基础扎实而稳拿高分的面貌。
考研,既要注意专业基础的夯实,也要兼顾攻克难题的能力,不是靠着几个模板,几个定律就包打天下。而凯程教育既严格要求,扎实训练,精准辅导,同时发掘学生的各个科目的潜力,从而实现考试分数的最大化。
重点知识串讲,答题方法和技巧,查漏补缺.
\end{document}