\documentclass[a4paper]{ctexart}
\usepackage{amsmath,amsthm,amsfonts,amssymb,bm,mathrsfs,wasysym}
\usepackage[top=2cm,bottom=2cm,left=1cm,right=1cm]{geometry}
\renewcommand{\baselinestretch}{1.5}
\usepackage{exscale} 
\usepackage{relsize}
%\usepackage{fourier} 
\begin{document}
\noindent
其实不少朋友问我的问题基本是:考北大数学怎么备考or学数分怎么学\\
我其实在老论坛发了不少帖子讲述我的看法(因为没实战考过,谈不上经验)\\
1.我学好数分无非是一点天赋+过人的勤奋\\
你可以想象一下,数学系的人有老师教,周围还有数学圈子,成年在数学氛围里浸泡(只要他不是在混),好歹有些感觉吧.我一工科生,谈不上有超越数学同仁们的IQ,如果你不付出更多的努力,凭什么超出数学系平均水平呢?\\
拿出你的时间余量.你靠自学,还是业余时间,你只要拿出更多的余量来独立思考.时间对大家是平等的,数学系因为专业限制一年可能会学五六门或以上数学课,每门课时间有限.当我可以一年内只看数分,反复看,想和做.这个道理很简单:你和一高手pk,一对一,你肯定打不过对方,那你就2打1,3打1,……,此我所谓自学数学的野蛮成长.\\
最初靠蛮力,经验和教训多了,加上自己经常系统性地总结(记些心得笔记),平时看书多留心,时间长了也会慢慢摸索出来适合自己的路来.(如果你细心观察我写的那些北大数分解答,就能看出我掌握能力的提高,就是这个道理)\\
这个道理对那些向往名校数学系的数学系出身朋友们,也是一样的.你只有拿出更多的时间和精力才能与名校数学系的高手们缩小差距,勤能补拙如是也.除此之外,我不相信有任何捷径.\\
至于我解答北大数分试题所参考的书,有两本\\
裴礼文的习题集 + 谢惠民的讲义\\
习题集有这两本考北大足矣,足矣!\\
今年(2011)北大数分考的相当惨烈(难道今年没调分?),最难的题应该是第9题证明极限存在那题吧.那个关键的Carleman不等式,谢惠民的讲义下册例题中就有啊.当然,本题不排除有简洁巧妙的解法存在.但考场游戏规则是:在规定时间内,最大化你的分数(很像拉格朗日乘数法吧).就像打仗一样,只听说宁肯仗打的不好看,但求速战速决;从没听说为了仗打的好看些,而把全军拖入旷日持久战的.讲数学简洁巧妙美之类的,等考上北大再讲也不迟啊.\\
最后,还要声明下.我不做系统题好多年了,也很少系统看数学了(言下之意,水平滑坡严重),偶尔也回论坛看看.但实在是尚无多余精力帮助那些考研或学数学的朋友们了.求助那些身在北大的网友们才是考北大正道.\\
祝所有关心我的新老朋友们心想事成,一帆风顺!\\
\end{document}